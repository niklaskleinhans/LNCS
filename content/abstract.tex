
\begin{abstract}
  Maschinen werden bis zum kleinsten Bauteil immer intelligenter und ermöglichen es Daten unterschiedlicher Struktur und Komplexität in großen Mengen aufzunehmen. 
  Im industriellen Umfeld wird dabei von \enquote{Industrie 4.0} gesprochen. 
  Der Begriff ummantelt die Beschreibung einer neuen Industriegeneration, in der Systeme durch technologische Weiterentwicklungen intelligent werden und miteinander vernetzt sind. 
  Ein Ergebnis dieser Intelligenz ist die Maschinenüberwachen. 
  Die Maschinenen sollen mithilfe von Livedatenanalyse überwacht und dadurch der aktuellen oder auch zukünftige Maschinenzustand ermittelt werden. 
  Um diese Intelligenz zu erreichen werden unteranderem Sensordaten aufgenommen und anschließend zur Analyse weiterverarbeitet.
  Die dadurch entstehenden Datenbilder lassen sich oft nicht mehr durch einfache, beispielsweise lineare, Analysmodelle abbilden.
  Die Anforderung an die Sensordatenanalyse im Maschinellenumfeld sind vorallem eine schnelle Datenverarbeitung mittels geringer Rechenkapazität.
  Um dies zu gewährleisten wird versucht die komplexen Sensordaten so aufzubereiten, dass diese weiterehin durch weniger komplexe Analyseverfahren verarbeitet werden können.
  In dieser Arbeit werden Verfahren vorgestellt die mithilfe von Feature Extraktion diese Anforderungen erfüllen.
  Dabei steht der Begriff \enquote{Feature} für ein Merkmal, wodurch sich Daten unterscheiden lassen.
  Diese werden extrahiert und bilden die Grundlage zur Datendifferenzierung. 
  Es werden die Herausforderungen in der Sensordatenanalyse, speziell in der Livedatenanalyse, beschrieben, Algorithmen vorgestellt und bezüglich ihrer Korrektheit und Laufzeit diskutiert.

\end{abstract}