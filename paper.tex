% This template has been tested with LLNCS DOCUMENT CLASS -- version 2.20 (24-JUN-2015)

%"runningheads" enables:
%  - page number on page 2 onwards
%  - title/authors on even/odd pages
%This is good for other readers to enable proper archiving among other papers and pointing to
%content. Even if the title page states the title, when printed and stored in a folder, when
%blindly opening the folder, one could hit not the title page, but an arbitrary page. Therefore,
%it is good to have title printed on the pages, too.
\documentclass[runningheads,a4paper]{llncs}[2015/06/24]

%cmap has to be loaded before any font package (such as cfr-lm)
\usepackage{cmap}
\usepackage[T1]{fontenc}

\usepackage{graphicx}

%some custom math stuff
\usepackage{amsmath}
\usepackage{amssymb}

% für neue deutsche Rechtschreibung
\usepackage[english,ngerman]{babel}

% für englische Rechtschreibung
%Even though `american`, `english` and `USenglish` are synonyms for babel package (according to https://tex.stackexchange.com/questions/12775/babel-english-american-usenglish), the llncs document class is prepared to avoid the overriding of certain names (such as "Abstract." -> "Abstract" or "Fig." -> "Figure") when using `english`, but not when using the other 2.
%english has to go last to set it as default language
%\usepackage[ngerman,english]{babel}

%Eingabeformat UTF-8
\usepackage[utf8]{inputenc}

%Hint by http://tex.stackexchange.com/a/321066/9075 -> enable "= as dashes
\addto\extrasenglish{\languageshorthands{ngerman}\useshorthands{"}}

%cfr-lm is preferred over lmodern. Reasoning at http://tex.stackexchange.com/a/247543/9075
\usepackage[%
rm={oldstyle=false,proportional=true},%
sf={oldstyle=false,proportional=true},%
tt={oldstyle=false,proportional=true,variable=true},%
qt=false%
]{cfr-lm}
%
%if more space is needed, exchange cfr-lm by mathptmx

\graphicspath{{graphics/}}

%Tweaks by IPVS/AS
\usepackage{lncs_as}

%for demonstration purposes only
\usepackage[math]{blindtext}

%Sorts the citations in the brackets
%It also allows \cite{refa, refb}. Otherwise, the document does not compile.
%  Error message: "White space in argument"
\usepackage{cite}


%% If you need packages for other papers,
%% START COPYING HERE
%% COPY ALSO cmap and fontenc from lines 10 to 12

%extended enumerate, such as \begin{compactenum}
\usepackage{paralist}

%put figures inside a text
%\usepackage{picins}
%use
%\piccaptioninside
%\piccaption{...}
%\parpic[r]{\includegraphics ...}
%Text...

%for easy quotations: \enquote{text}
\usepackage{csquotes}

%enable margin kerning
\usepackage{microtype}

%tweak \url{...}
\usepackage{url}
%\urlstyle{same}
%improve wrapping of URLs - hint by http://tex.stackexchange.com/a/10419/9075
\makeatletter
\g@addto@macro{\UrlBreaks}{\UrlOrds}
\makeatother
%nicer // - solution by http://tex.stackexchange.com/a/98470/9075
%DO NOT ACTIVATE -> prevents line breaks
%\makeatletter
%\def\Url@twoslashes{\mathchar`\/\@ifnextchar/{\kern-.2em}{}}
%\g@addto@macro\UrlSpecials{\do\/{\Url@twoslashes}}
%\makeatother

%diagonal lines in a table - http://tex.stackexchange.com/questions/17745/diagonal-lines-in-table-cell
%slashbox is not available in texlive (due to licensing) and also gives bad results. This, we use diagbox
%\usepackage{diagbox}

%required for pdfcomment later
\usepackage[hyperref,svgnames]{xcolor}

\usepackage{listings}
\lstloadlanguages{java}
\lstset{language=java,numbers=left,captionpos=b}


%enable nice comments
%this also loads hyperref
\usepackage{pdfcomment}
%enable hyperref without colors and without bookmarks
\hypersetup{hidelinks,
   colorlinks=true,
   allcolors=black,
   pdfstartview=Fit,
   breaklinks=true}
%enables correct jumping to figures when referencing
\usepackage[all]{hypcap}

\newcommand{\commentontext}[2]{\colorbox{yellow!60}{#1}\pdfcomment[color={0.234 0.867 0.211},hoffset=-6pt,voffset=10pt,opacity=0.5]{#2}}
\newcommand{\commentatside}[1]{\pdfcomment[color={0.045 0.278 0.643},icon=Note]{#1}}

%compatibality with packages todo, easy-todo, todonotes
\newcommand{\todo}[1]{\commentatside{#1}}
%compatiblity with package fixmetodonotes
\newcommand{\TODO}[1]{\commentatside{#1}}

%enable \cref{...} and \Cref{...} instead of \ref: Type of reference included in the link

\usepackage[capitalise,nameinlink,ngerman]{cleveref}
%\usepackage[capitalise,nameinlink,english]{cleveref}
%Nice formats for \cref - only for English texts
%\crefname{section}{Sect.}{Sect.}
%\Crefname{section}{Section}{Sections}

\usepackage{xspace}
%\newcommand{\eg}{e.\,g.\xspace}
%\newcommand{\ie}{i.\,e.\xspace}
\newcommand{\eg}{e.\,g.,\ }
\newcommand{\ie}{i.\,e.,\ }

%introduce \powerset - hint by http://matheplanet.com/matheplanet/nuke/html/viewtopic.php?topic=136492&post_id=997377
\DeclareFontFamily{U}{MnSymbolC}{}
\DeclareSymbolFont{MnSyC}{U}{MnSymbolC}{m}{n}
\DeclareFontShape{U}{MnSymbolC}{m}{n}{
    <-6>  MnSymbolC5
   <6-7>  MnSymbolC6
   <7-8>  MnSymbolC7
   <8-9>  MnSymbolC8
   <9-10> MnSymbolC9
  <10-12> MnSymbolC10
  <12->   MnSymbolC12%
}{}
\DeclareMathSymbol{\powerset}{\mathord}{MnSyC}{180}

% correct bad hyphenation here
\hyphenation{op-tical net-works semi-conduc-tor}

%% END COPYING HERE


\begin{document}

\title{Feature-Extraktion für Sensordaten zur Maschinenüberwachung}
%If Title is too long, use \titlerunning
%\titlerunning{Short Title}

\author{Niklas Kleinhans}

\supervisor{Mathias Mormul}

\seminar{Advanced Topics in Data Management}

\semester{WS 2018/2019}

\abgabedatum{Stuttgart, 05.11.2018}

\institute{}

\frontpagede % creates the frontpage (in German)
%\frontpageen % creates the frontpage (in English)

\thispagestyle{empty}
\cleardoublepage

\maketitle


\begin{abstract}
  Maschinen werden bis zum kleinsten Bauteil immer intelligenter und ermöglichen es Daten unterschiedlicher Struktur und Komplexität in großen Mengen aufzunehmen. 
  Im industriellen Umfeld wird dabei von \enquote{Industrie 4.0} gesprochen. 
  Der Begriff ummantelt die Beschreibung einer neuen Industriegeneration, in der Systeme durch technologische Weiterentwicklungen intelligent werden und miteinander vernetzt sind. 
  Ein Ergebnis dieser Intelligenz ist die Maschinenüberwachen. 
  Die Maschinenen sollen mithilfe von Livedatenanalyse überwacht und dadurch der aktuellen oder auch zukünftige Maschinenzustand ermittelt werden. 
  Um diese Intelligenz zu erreichen werden unteranderem Sensordaten aufgenommen und anschließend zur Analyse weiterverarbeitet.
  Die dadurch entstehenden Datenbilder lassen sich oft nicht mehr durch einfache, beispielsweise lineare, Analysmodelle abbilden.
  Die Anforderung an die Sensordatenanalyse im Maschinellenumfeld sind vorallem eine schnelle Datenverarbeitung mittels geringer Rechenkapazität.
  Um dies zu gewährleisten wird versucht die komplexen Sensordaten so aufzubereiten, dass diese weiterehin durch weniger komplexe Analyseverfahren verarbeitet werden können.
  In dieser Arbeit werden Verfahren vorgestellt die mithilfe von Feature Extraktion diese Anforderungen erfüllen.
  Dabei steht der Begriff \enquote{Feature} für ein Merkmal, wodurch sich Daten unterscheiden lassen.
  Diese werden extrahiert und bilden die Grundlage zur Datendifferenzierung. 
  Es werden die Herausforderungen in der Sensordatenanalyse, speziell in der Livedatenanalyse, beschrieben, Algorithmen vorgestellt und bezüglich ihrer Korrektheit und Laufzeit diskutiert.

\end{abstract}
\begin{keywords}
Maschinenüberwachung, Feature Extraktion, Maschinelles Lernen, Sensordaten, Livedaten, Livedatenanalyse
\end{keywords}

\section{Einleitung}\label{kap:einleitung}

%Ein großer Bereich der vierten Industriegeneration ist das \enquote{Internet der Dinge} oder im Englischen \enquote{Internet of Things (IoT)}. 
%Darunter versteht man ein erweitertes Konzept des aktuel bestehenden Internets. 
%Es findet nicht mehr nur ein reiner Daten-/Informationsaustausch mittels unterschiedlicher Medientypen statt, sondern große Systeme bis hin zu kleinsten Komponenten werden Vernetzt und bilde eine Kommunikationsschicht~\cite{Jaspindustrie}.
%Durch diese Kommunikation entstehen Datenflüsse die einerseits in großen Rechenzentren gespeichert werden können und andererseits bieten sie eine Grundlage zur Livedatenanalyse.

Maschinen bestehen oft aus mehreren Komponenten und sind anschließend Teil eines großen Systems.
Um zu gewährleisten, dass das System korrekt funktioniert und auch möglichst selten ausfällt, werden Daten über Kontrollsysteme und Sensoren jeder einzelnen Komponente aufgenommen.
In einem großen System fallen durch kontinuierliche Datenaufnahme große Datenmengen an.
Um schon preventiv Maschinenausfälle zu vermeiden werden diese Daten oft an Überwachungssysteme übertragen um anschließend mithilfe von Schwellwertüberschreitungen den Systemzustand darzustellen.
Diese Datenbilder werden in Form von Diagrammen, Ampelsystemen, oder ähnlichem visualisiert und für den Menschen lesbar dargestellt.
Es ist die Rede von \enquote{Condition-Monitoring}.

Um noch mehr Informationen aus diesen Daten zu erhalten werden unterschiedliche Analyseverfahren angewendet um Anomalien festzustellen und auf diese anschließend zu reagieren.
Die Herausforderung dabei ist der kontinuierliche Datenfluss.
Die Daten müssen, ähnlich zur Livedatenanalyse, direkt verarbeitet werden.
Schwellwertüberschreitungen können diese Livedaten leicht verarbeiten, da jeweils nur ein Datenpunkt betrachtet werden muss.
Nähere Zusammenhänge innerhalb eines Datenpakets oder sogar zwischen mehrere Datenpaketen zu erkennen ist Rechenaufwändiger.
Betrachtt man mehere Daten innerhalb eines Zeitfensters können diese Datenpunkte mittels Approximationen wie in Abbilding\ \ref{fig:FFETimeSeries} (a), (b) und (c) verglichen werden. 
Die Zeitreihendaten sehen in diesem Fall für Menschen nahezu gleich aus. 
Dennoch kann durch Anwenden von Analysverfahren ein gewisser Unterschied erkannt werden.
In Abbildung\ \ref{fig:FFETimeSeries} (d), wurden die Zeitreihen in einem Merkmalsraum, im englischen \enquote{Feature Space}, dargestellt. 
Zeireihe (a) und (c) sind weiterhin sehr ähnlich.
Zeireihe (b) unterscheidet sich etwas im Vergleich zu (a) und (c).

\begin{figure}
  \centering
  \includegraphics[width=1\textwidth]{FastFeatureExtractionTimeSeries.png}
  \caption{Drei standard ploynom Approximationen mittes least-squared. Die Drei Approximationen werden in Abbildung (d) in einem Merkmalsraum dargestellt~\cite{gensler2015fast}.}
  \label{fig:FFETimeSeries}
\end{figure}


Dieser Merkmalsraum wird durch das Analysieren der Daten auf gewisse Merkmale, im englischen \enquote{Features}, erstellt.
Der Merkmalsraum bietet eine weitere Möglichkeit Daten miteinander zu vergleichen, Anomalien zu erkennen oder die Daten zu klassifizieren. 
Um solche Verfahren anwenden zu können müssen diese Merkmale zuvor extrahiert werden. 
Es ist die Rede von \enquote{Feature Extraktion}.

In diesem Fall handelt es sich um eine polynomielle Approximation der Zeitreihendaten. 
Diese muss in in jedem Zeitschritt für den gewählt Zeitraum neu berechnet werden. 
Dies kann je nach Dimension des zu approximierenden Polynoms sehr Rechenaufwändig und komplex werden.
Bei einer hohen Abtastrate und einer großen Anzahl an Sensoren entstehen rießige Datenmengen, die unter harten Laufzeitbedingungen analysiert werden müssen.
Ein weiterer Vorteil der \enquote{Feature Extraktion} ist das Reduzieren der Komplexität der Daten. Die Daten müssen nicht mehr selbst über polynomielle Modelle analysiert werden, sondern die extrahierten Merkmalen können als Eingabe für die Analyseverfahren dienen. Somit können komplexe polynomielle Datenbilder mithilfe von beispielsweise linearer Verfahren analysiert werden.

In dieser Arbeit wird zu Beginn in Kapitel \ref{kap:grundlagen} die grundlegende Struktur von Sensordaten beschrieben, der Zusammenhang mit Livedaten diskutiert und anschließend die damit verbundene Verwendung von \enquote{Feature Extraktions} Verfahren besprochen. 
In Kapitel \ref{kap:featureextraktion} werden die Randbedingungen und Herausforderungen an Algorithmen zur kontinuierlichen Livedatenanalyse besprochen und Ansätze sowie Algorithmen vorgestellt, die sich mit desem Problemen auseinander setzen.
\section{Verwandte Arbeiten}
Viele Arbeiten  beschäftigen sich damit Sensordaten mit Hilfe von analytischen Methoden zu verarbeiten. 
Ein Beispiel ist die Arbeit \enquote{Using Machine Learning on Sensor Data} von Alexandra Moraru et al.~\cite{moraru2010using}. 
In dieser Arbeit wird die Anzahl von Mitarbeitern im Büro anhand von Sensordaten vorhergesagt.
Dabei werden klassische Klassifikations- und Regressionsverfahren angewendet und validiert. 
Die Verfahren werden auf einen Trainingsdatensatz trainiert und anschließend auf weitere Testdaten angewendet. 
Die Merkmale entsprechen den verwendeten Parameterwerten. 
In diesem Fall ist die Anzahl dieser Parameter nicht sehr groß. 
Da die Daten nachverarbeitet werden und nicht unter die starken Speicher- und Laufzeitbedingungen Fallen, ist in diesem Fall keine Komplexitätsreduktion notwendig.

Bei komplexeren Problemen, wie bei der Maschinenüberwachung, werden die Merkmale vorerst extrahiert, um die Daten vergleichen zu können und um die Komplexität zu reduzieren.  
Diese Merkmale können neben einfachen Schwellwertüberschreitungen auch beispielsweise gewisse Datenmuster sein. 
Andre Gensler, Thiemo Gruber und Bernhard Sick beschreiben in ihrer Arbeit \enquote{Fast Feature Extraction For Time Series Analysis Using Least-Squares Approximations with Orthogonals Basis Functions} ein Verfahren, mit welchen sie diese Merkmale, unter harter Laufzeit und Speicheranforderungen, ermitteln~\cite{gensler2015fast}. 

Die Analyseverfahren benötigen oft sehr viel Rechenleistung da die Probleme, welche einer nicht linearen Komplexität entsprechen, sehr aufwändig zu Berechnen sind. 
Man spricht dabei von Höherdimensionalen Problemen. 
Fabian Mörchen beschreibt in seiner Arbeit \enquote{Time series feature extraction for data mining using DWT and DFT} eine Methode um die Dimension von Zeitreihendaten zu reduzieren~\cite{morchen2003time}. 
Er stellt die Zeitreihendaten mittels Fourier Transformation periodisch dar und verwendet die beschreibenden Parameter als Merkmale.
Um nicht alle Parameter als Merkmale zu verwenden, beschreibt er eine von ihm definierte Aggregatfunktion.
Aggregationsfunktionen sind Funktionen, die Eigenschaften von Daten zusammenfassen.  
In diesem Fall werden die Informationshaltigsten Parameter für die Fourier-Darstellung ausgewählt.
Aggregationsfunktionen sind Funktionen, die Eigenschaften von Daten zusammenfassen.  
Er erzeugt damit eine messbare Darstellung der Daten in einem reduzierten Rahmen um damit den Berechnungsaufwand zur Analyse zu reduzieren.

Weitere Verfahren werden in dem Artikel \enquote{Time Series Feature Extraction} von Michael A. Trovero und Michael J. Leonard. vorgestellt~\cite{Leonard2018}. 
Sie konzentrieren sich dabei auf verschiedene Möglichkeiten die Zeitreihen in Merkmale zu Zerlegen.
Beispielsweise nutzen sie additive oder multiplikative Zerlegungsverfahren, welche die Daten auf eine feste Merkmalsmenge reduzieren.
Es wird in ihrer Arbeit auch ein Verfahren zur Motivfindung beschrieben. Die Anzahl an gefundenen Motiven ergibt die Anzahl an messbaren Merkmalen.

Dominique Gay, Romain Guigoures, Marc Boulle, and Fabrice Clerot stellen in ihrer Arbeit \enquote{Feature Extraction over Multiple Representations for Time Series Classification} eine Möglichkeit vor, die Daten anhand von einem Koordinatengitter zu beschreiben~\cite{Gay2013}.
Dabei werden die klassifizierten Trainingsdaten in ein Dreidimensionalen Raum dargestellt und anhand der Anzahl in einem Gitter vorkommenden Datenpunkten bewertet.
Dabei werden stark frequente Gitter als Merkmale extrahiert. 
Als Ergebnis können neue nicht klassifizierte Daten anhand der Gitterzuordnung in die jeweiligen Klassen zugeordnet werden. 
\section{Grundlagen}\label{kap:grundlagen}
In diesem Kapitel werden einige Grundlagen für die in Kapitel \ref{kap:featureextraktion} diskutierten Algorithmen besprochen.
\subsection{Was sind Sensordaten}\label{kap:sensordaten}

Um den Ablauf einer Maschine zu koordinieren und den aktuellen Zustand zu Überwachen werden oft Sensoren an den Maschinen angebracht. 
Diese Daten werden zu definierten Taktzeiten aufgenommen und weiterverarbeitet. 
Es kann sich dabei um einfache Kontaktsensoren handeln, mit einer begrenzten Anzahl an Zuständen oder Sensoren mit einem reellen Zustandsbereich wie Umgebungssensoren (Luftdruck-, Temperatursensor, etc.). 
In Abbildung\ \ref{fig:datasetoffice} sind Sensordaten in Form eines Datenplots dargestellt. 
Es handelt sich um Licht-, Temperatur- und Drucksensordarten aus einem Büro im Zeitraum eines Arbeitstages. 
Abgebildet sind die Sensordaten zwischen $6:00$ Uhr und $19:00$Uhr.
\begin{figure}
    \centering
    \includegraphics[width=.8\textwidth]{datasetOffice.png}
    \caption{Sensordatenabweichung anhand von Licht, Temperatur und Luftfeuchtigkeit im Buro innerhalb von einem Arbeitstag~\cite{moraru2010using}}
    \label{fig:datasetoffice}
\end{figure}

Um solche Sensordaten mathematisch weiterverarbeiten zu können, werden sie in Vektorform gebracht.
Ein Datensatz zum Zeitpunkt $6:00$ Uhr mit den Werten Licht = $0$, Temperatur=$22$ und Luftfeuchtigkeit = $25$ könnte dargestellt werden als:
\begin{equation}
    x_1 =
  \begin{pmatrix}
      0\\
      22\\
      25
  \end{pmatrix}
\end{equation}

Ein Vektor enthält somit einen Seonsordatensatz zu einem Zeitpunkt. Um Zeiträume darzustellen werden die Daten in eine Matrixform gebracht. Beispielsweise erhält man bei einer Stündlichen Abtastrate und einem Zeitraum von $8:00$ Uhr bis $17:00$Uhr folgende Matrix:
\begin{equation}
    x_{1,10}
    \begin{pmatrix}
       38 & 40 & 46 & 24 & 60 & 58 & 51 & 44 & 40 & 36\\
       20 & 22 & 22 & 24 & 24 & 24 & 20 & 18 & 20 & 17\\
       25 & 28 & 28 & 28 & 29 & 30 & 20 & 27 & 29 & 26
    \end{pmatrix}
\end{equation}

Diese mathematische darstellung ist nur ein Beispiel und kann beliebig strukturiert werden.

Die erzeugte Matrix kann anschließend als Eingabe für Analysealgorithmen dienen.
Schon bei diesen geringen Datenmengen entsteht eine $K x N $ große Matrix, wobei $K=$ Anzahl der Paramter (Sensoren) und $N=$ Anzahl der Daten.

Sollen auch noch Daten Raumübergreifend analysiert werden, können diese in Form eines Tensors dargestellt werden. Bei $M$ vielen Räumen entsteht ein $K x N x M$ großer Tensor.
Schon anhand dieses simplen Beispiels wird die Datenmenge und Komplexität der Daten ersichtlich.
In der Maschinenüberwachung entstehen dadurch schnell Datensätze im Millionenbereichen und da jede Komponente einer Maschine oft mit mehereren Sensoren ausgestattet ist, entstehen riesige Tensoren.

Neben klassischen Regressionsverfahren zur Datenanalyse, welche oft für Anomliedetektionen verwendet werden, gibt es auch verschiedene Klassifikationsverfahren. Dazu werden den Datensätzen manuell oder automatisiert Klassen hinzugefügt.

Mathematisch dargestellt erhalten wir dann einen Datensatz zum Zeitpunkt $t_1$ in Form eines Tupels $\tau_1=(t_1,x_1)$. Diesem Tupel wird abhängig von den verwendeten Verfahren eine Klasse $y_1$ zugewiesen~\cite{gay2013feature}. Für den Zeitraum $(t_1,...,t_n)$ mit $n \in \mathbb{N}$ vielen Daten erhalten wir den Datensatz 
\begin{equation}
  D=\{ (\tau_1,y_1), ... , (\tau_n,y_n) \}
  \label{equ:trainingset}
\end{equation}
Das Ziel kann es sein das Tupel $(t_{n+1},y_{n+1})$ voherzusagen.

Es werden auch nicht nur feste Zeiträume betrachtet. 
Durch dauerhaft laufende Maschinen entstehen kontinuierliche Datenströme.
Daraus folgt ein kontiniuerlich wachsender Datenbestand.
Um ressourcenschonend und möglichst in Echtzeit die Daten zu analysieren, werden harte Speicher- und Laufzeitbedinungen an Analysealgorithmen gestellt.


\subsection{Feature Extraktion}\label{kap:featureextraktionuebersicht}
In Kapitel \ref{kap:sensordaten} wurde ein Datensatz durch seine Muster und Merkmale beschrieben. Merkmale können einfache Parameter, wie Schwellwertüberschreitungen sein. In Komplexeren Datenstrukturen, können die Daten anhand von Verlaufsmustern unterschieden werden. 

Um Algorithmen zu trainieren die Daten durch diese Merkmale und Muster zu unterscheiden, gibt es zwei wesentliche herangehensweisen. Es können die Merkmale dem Algorithmus vorgegeben und auf diese Merkmale trainiert werden, oder es werden Verfahren angewendet um diese Merkmale zu extrahieren. Es ist die Rede von \enquote{Feature Extraktion}. Ein \enquote{Feature} ist eines dieser Merkmale, wodurch Daten in einem Datensatz voneinander unterschieden werden können. 

Dabei ist das Ziel die Features so zu wählen und zu parametrisieren, dass das der ermittelte Wert dem tatsächlichen Wert möglichst ähnelt. Betrachten wir die Gleichen \ref{equ:trainingset} und die daraus ermittelte Funktion $p(x_i)$, welche das Ergebnis des gewählten Algorithmus und des Trainingsdatensatzes ist. Es soll versucht werden den Fehlerunterschied

\begin{equation}
  \sigma = p(x_i)-y_i
\end{equation}

möglichst zu reduzieren~\cite{gensler2015fast}. Konkret wird meist die Summe der Quadratischefehler
\begin{equation}
  \sigma^2 = \frac{1}{N} \sum_{i=0}^{K}(p(x_i)-y_i)^2
\end{equation}
versucht zu reduzieren.

\begin{center}
  - TODO - Ein konkretes Beispiel für Feature Extraktion raussuchen und daran erkären. Es muss auch noch die Notwendigkeit und die herangehensweise von Dimensionsreduzierung erläutert werden.
\end{center}

\begin{itemize}
  \item Kurze ML einleitung mit erklärung zur Feature-Extraktion
  \item Feature extraction vs Feature selection
  \item \enquote{extrahieren von Merkmalen, wodurch die Daten in einem Datensatz voneinander unterschieden werden können}
\end{itemize}
\section{Feature Extraktion bei kontinuierlichen Livedaten}\label{kap:featureextraktion}

Die Sensordaten zur Überwachung von Maschinen erzeugen einen kontinuierlichen Datenfluss. Wie in den voherigen Kaptiteln beschrieben, stellt diese Eigenschaft eine hohe Anforderung an Lernalgorithmen. Einige Lernalgorithmen verwenden das Konzept des \enquote{Lazy Learnings}, im Deutschen \enquote{träges Lernen}. Bei diesen Lernalgorithmen wird das Modell auf Anfrage erstellt. Das bedeutet, jede Eingabe startet eine neue Modellbildung und fordet daher in kürzester Zeit eine hohe Rechenleistungen. Beim \enquote{Eeager Learning}, im Deutschen \enquote{Eifriges Lernen}, hingegen wird schon im Vorfeld mithilfe von Trainingsdaten ein Model bereitgestellt. Der Ressourcenverbrauch bei einer Anfrage von neuen Daten ist dann sehr gering~\cite{Gay2013}. 

Bei den kontinuierlichen Livedaten in der Maschinenüberwachung stehen die Daten meist auch zeitlich in korrelation. Daher ist die Rede von Zeitreihen Daten. Die Reihenfolge dieser Daten spielt bei der Analyse eine große Rolle. Zeitreihendatensätze werden mathematisch wie in dem Datensatz \ref{equ:trainingset} dargestellt. Eines der Ziele für Lernalgorithmen kann es sein aus dem Datensatz \ref{equ:trainingset} die Daten 

\begin{equation}
  \tau_{n+1}, \tau_{n+2}, ...
  \label{equ:predictionset}
\end{equation}
voherzusagen.

Um die Komplexität der Zeitreihendatensätze zu reduzieren wird die Dimension, entstehend durch die Parameter, versucht möglichst auf die notwendigen Merkmale zu reduzieren. Zeitreihendaten aus Sensoren sind oft hoch korrelierend, was zu einer sehr großen Datenredunanz führt. 
Eine bewerte Technik um Datenmengen aus Sensordaten fester größe darzustellen ist die \enquote{Fourier Transforamtion}. Dabei werden die Signale auf einen Frequenzbereich abgebildet und diese Abbildung mittels Koeffizienten dargestellt. Da es sich bei den kontinuierlichen Sensordaten meist um keinen vollständigen Datenbestand handelt, sondern nur Datenauszüge bietet sich die \enquote{Diskrete Fourier Transformation} an.

\begin{equation}
  F_j = \frac{1}{N} \sum_{k=0}^{N-1} f_k W_N^{-kj}  \text{\ \ mit \ \ } W_N = e^{\frac{2\pi i}{N}}
  \label{equ:fourier}
\end{equation}

Durch die Diskrete Fourier Transformation in Formel\ \ref{equ:fourier} wird der Zeitreihenabschnitt in einer periodische Funktion durch Koeffizienten dargestellt~\cite{Butz2012}. Um die zu analysierdenen Merkmale zu reduzieren ist die Idee,dass nicht alle Koeefizienten als Parameter verwenden werden, sondern nur eine Auswahl von diesen. Bekannte Methoden sind dabei entweder die ersten $k$ Koeffizienten zu verwenden. Man speichert quasi nur eine grobe Skizze der Kurven ab. Die ersten Koeffizienten abzuspeichern, behält die tieferen Frequenzen und ist eine sehr naive herangehensweise. Eine deutlich bessere Methode wäre es die größten Koeffizienten zu verwenden. Diese sind aber sehr aufwändig zu berechnen~\cite{morchen2003time}. Fabian Mörchen stellt dagegen eine Methode in seiner Arbeit \enquote{Time series feature extraction for data mining using DWT and DFT} vor, in der er eine Aggregatsfunktion verwendet, welche die Bedeutung der Koeffizienten misst. Es ist dadurch möglich mit seiner Aggregatsfunktion:

\begin{equation}
  J_k^1(mean(c_j^2), C)
  \label{equ:aggregatefunction}
\end{equation}

eine definierte Menge $k$ an Koeffizienten als Merkmale für Lernalgorithmen als Eingabe zu geben. $k$ entspricht dann der Dimension der zu verarbeitenden Datenmenge. 

Eine weitere Arbeit von Dominique Gay et al. \enquote{Feature Extraction over Multiple Representations for Time Series Calssification} stellt ein Verfahren vor in dem sie Zeitreihendaten so vorverarbeiten, dass in dem Verfahren extrahierte Merkmale die neuen Parameter fester Größe bilden. Dies geschieht in einem drei Schritteverfahren:

\begin{enumerate}
  \item Der Datensatz wird in mehrere Datenrepräsentationen transformiert
  \item Auf jede Repräsentation wird ein Co-Clustering Verfahren angewendet
  \item Aus jeder Repräsentation wird eine Menge an Merkmalen erstellt und daraus ein neuer Datensatz generiert
\end{enumerate}

Für den ersten Schritt schlagen sie verschiedene Transformationsverfahren vor. Beispielsweise Ableitungen oder comulatives Integrieren. In einem Beispiel sind die Vorteile einer solchen Vorverarbeitung erkennbar. In Abbildung\ \ref{fig:derivative} ist in (a) ein zwei Klassen ARSim Datensatz zu sehen. Dieser ist unverarbeitet und die Klassen sind farblich Markiert. Die Klassen sind in (a) nur sehr schwer separierbar. In (b) wurde der selbe Datensatz durch zweifaches Ableiten transformiert und wieder in einem Datenplot und farblicher markierung dargetsellt. Durch die Transformation sind die Klassen schon deutlicher erkennbar.

\begin{figure}
  \centering
  \includegraphics[width=1\textwidth]{plotAbleitung.png}
  \caption{In (a) ein unverarbeiteter original geplotteter Datensatz und in (b) der gleiche durch zweifache Ablteiungen transformierter Datensatz~\cite{Gay2013}.}
  \label{fig:derivative}
\end{figure}

Der transformierte Datensatz bilde somit eine bessere Grundlage um mithilfe von Klassifizierungsalgorithmen die Daten zu Differenzieren. 
Der zweite Schritt ist das Co-Clustering. 
Dabei wird das Clustering als eine Vorverarbeitung für nachfolgende Lernalgorithmen verwendet. 
Die Idee ist es ähnliche Daten zu gruppieren und lokale Muster hervorzuheben~\cite{gay2013feature}. Dabei stellen sie die Kurven in einer Menge $(X,Y)$ dar und fügen jeder dieser Mengen eine Klasse $Cid$ hinzu um sie der jeweiligen Kurve zuzuordnen. Es entsteht eine dreidimensinale Darstellung der Punkte. Das Ganze wird in eine dreidimensionale Gitterstruktur gebracht. Das Endziel ist es Kurven- und Intervallcluster zu erhalten die Anschließend als Merkmalsgrundlage dienen. 

Erreicht wird das durch das Anwenden des \enquote{Khiops Coclustering}\cite{boulle2012functional}. Dabei wird das optimale Gitterfeld durch die Optimierung des Bayes'schen Kriteriums, der sogenannen Kosten ermittelt. 
\begin{equation}
  cost(M) = -log(p(M) \times p(D|M))
  \label{equ:Bayesian}
\end{equation}
Als resultat lassen sich die Kosten so interpretieren, dass bei niedrigen Kosten eine hohe Kompression der Daten $D$ auf das Modell $M$ herrscht. Wobei das Modell in diesem Fall das optimale Gitterfeld ist.

\begin{figure}
  \centering
  \includegraphics[width=1\textwidth]{coclustering.png}
  \caption{Ergebniss eines CoCluserings mit Khiops Coclustering~\cite{Gay2013}.}
  \label{fig:coclustering}
\end{figure}

Ein durchgeführtes Co-Clustering ist in Abbildung\ \ref{fig:coclustering} zu sehen. Dabei ist die dritte Dimension Farblich dargestellt. Als Ergebnis sind 25 Cluster von Kurven entstanden. 

Als letzten Schritten müssen noch die Merkmale extrahiert und ein Datensatz generiert werden. Es werden dre Merkmale definiert:

\begin{enumerate}
  \item $K_C$ ein numerisches Merkmal, welches die Unähnlichkeitswahrscheinlichkeit zu allen Kurvenclustern angibt.
  \item Ein kategorisches Merkmal als index, welcher das nächste Kurvencluster angibt.
  \item $K_Y$ ein nummerisches Merkmal, welches die Anzahl an Punkten dieser Kurve in der jeweiligen Klasse angibt.
\end{enumerate}

Durch dieses Verfahren wird die Dimension der Livedaten mit Hilfe von \enquote{Feature Extraktion} auf Drei festgelegt und kann somit durch \enquote{Eager Learning} Algorithmen analysiert werden.

\section{Fazit}
Die Möglichkeiten Maschinen zu Überwachen und den aktuellen Maschinenzustand mittels Sensoren zu bestimmen werden immer größer. Nicht nur gegenwärtige Zustände, sondern auch vorhersagen von Zuständen sind mittlerweile möglich. Dadurch, dass schon kleinste Komponenten einer Maschine Sensordaten sammeln und diese zur Analyse bereitstellen, entsteht eine riesige Masse an Daten zur Weiterverarbeitung. Doch diese kontinuierlichen Datenflüsse stellen Lernalgorithmen an große Herausforderungen. 
Die Lernalgorithmen müssen unter harten Speicher- und Laufzeitanforderungen arbeiten und aus der Fülle an Daten die wichtigsten Informationen herausfiltern und diese auch möglichst noch für den Menschen verwendbar darstellen. 
Sensordaten fangen meist als einfache numerische Werte an, werden über die Anzahl an Parametern zu Vektoren zusammengefasst und ergeben über Vielfalt und Zeit meist sehr hochdimensionale Tensoren. In dieser Arbeit wurden Verfahren behandelt, welche diese Dimensionen für Lernalgorithmen durch Vorverarbeitung reduzieren.

Dabei werden \enquote{Feature Extraktions} Verfahren angewendet. Unter einem \enquote{Feature} versteht man ein Merkmal, wodurch sich Daten unterscheiden lassen. Verschiedene Ansätze versuchen die relevantesten Merkmale auszuwählen (Feature Selektion) oder durch Berechnungen neuer Merkmale zu extrahieren (Feature Extraktion). 

Einige dieser Ansätze bedienen sich bekannter Methoden, wodurch Datensätze mittels Transformationen parametrisiert werden und die Koeffizienten als Merkmale weiterverwendet werden können. Andere Ansätze Zerlegen die Daten durch additive oder multiplikative Methoden in verschiedene Informationshaltige Strukturen um aus diesen die Informationen besser extrahieren zu können. Ein Beispiel dafür die die Zerlegung in Trend und Saisonkomponenten.

Die Darstellung der Sensordaten als Kurven in einem Datenplot bietet auch visuelle Ansätze, indem die Daten Räumlich dargestellt werden und und dann durch Räumliche Zuordnungen klassifiziert und durch Merkmale unterschieden werden können. 

Andere Ansätze versuchen den Berechnungsaufwand so zu beschränken, dass zu Beginn die Berechnung unabhängig von der Größe des Datensatz ist und nur noch von der durch die Anzahl der Parameter bestimmten Dimension abhängt. Das anschließende Anwenden von \enquote{Feature Extraktion} liefert eine starke Laufzeiteinsparung.

Leider ist nie garantiert, dass durch die Extraktion ein Informationsverlust besteht und daher gibt es keine generelle Lösung für alle Datensätze. Die sogenannten Domänenexperten, die Personen welche sich Fachlich mit den Daten auskennen, sind vorerst Teil der Perfekten Analysekonfiguration. 



%%%%%%%%%%%%%%%%%%%%%%%%%%%%%%%%%%%%%%%%%%%%%%%%%%%%%%%%%%%%%%%%%%%%%%%%%%%%%%%

\bibliographystyle{splncs03}
\bibliography{paper}
%%\nocite{*}
Alle Links wurden zuletzt am 10.12.2018 geprüft.
\end{document}
