\section{Verwandte Arbeiten}
Viele Arbeiten  beschäftigen sich damit Sensordaten mit Hilfe von analytischen Methoden zu verarbeiten. 
Ein Beispiel ist die Arbeit \enquote{Using Machine Learning on Sensor Data} von Alexandra Moraru et al.~\cite{moraru2010using}. 
In dieser wird die Anzahl von Mitarbeitern im Büro anhand von Sensordaten vorhergesagt.
Dabei werden klassische Klassifikations- und Regressionsverfahren angewendet und validiert. 
Die Verfahren werden auf einen Trainingsdatensatz trainiert und anschließend auf weitere Datenpakete angewendet. 
Dabei wurden die Merkmale, an welchen die Anzahl der Mitarbeiter vorhergesagt werden sollen vordefiniert. 

Bei komplexeren Problemen, wie bei der Maschinenüberwachung, müssen die Merkmale oft erst gefunden werden. 
Diese Merkmale können neben einfachen Schwellwertüberschreitungen auch beispielsweise gewisse Datenmuster sein. 
Diese  Andre Gensler, Thiemo Gruber und Bernhard Sick beschreiben Verfahren in ihrer Arbeit \enquote{Fast Feature Extraction For Time Series Analysis Using Least-Squares Approximations with Orthogonals Basis Functions}, mit welchen sie diese Merkmale, unter harter Laufzeit und Speicheranforderungen, erkennen. 

Die Analyseverfahren benötigen oft sehr viel Rechenleistung da die Proleme, welche einer nicht linearen Komplexität entsprechen, sehr aufwändig zu Berechnen sind. 
Man spricht dabei von Höherdimensionalen Problemen. 
Fabian Mörchen beschreibt in seiner Arbeit \enquote{Time series feature extraction for data mining using DWT and DFT} eine Methode um die Dimension von Zeitreihendaten zu reduzieren. 
Er Optimiert dabei die Auswahl der Koeffizienten und reduziert damit den Berechnungsaufwand für Analyseverfahren.

\begin{center}
  -TODO- Die weiteren Referenzen zusammenfassen
\end{center}
